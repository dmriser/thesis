The three dimensional structure of protons and neutrons has been a subject of great interest since the early 1990s when advancements in theory produced the Generalized Parton Distribution (GPD) framework for understanding spatial distributions, and Transverse Momentum Dependent Parton Distribution Functions (TMD PDFs) for understanding quark momentum.  In this thesis, I present two analyses of the SIDIS process which can be related to TMDs.  I have used CLAS data from the E1-F run period, which recorded over 1 billion events of 5.498 GeV electrons colliding with a liquid hydrogen ($LH_2$) target.  First, I discuss analysis of the SIDIS cross section (fully differential in 5 variables) for both charged pions over a wide kinematic range.  This analysis observes a non-zero $\cos(\phi_h)$ and $\cos(2\phi_h)$ modulations, and is useful for phenomenological extractions of the Boer-Mulders and Cahn effect terms.  I then present results for the beam spin asymmetry (BSA), which is only sensitive to terms containing twist-three distribution functions.  The analysis, performed for $K^+$ mesons, observes a non-zero BSA and therefore implies that twist-three effects are not negligably small in the kinematics used for the study.  Unlike the large magnitude of the Sivers asymmetries observed at HERMES for $K^+$ mesons, the BSA reported here is of the same magnitude as the lighter $\pi^+$ meson. 

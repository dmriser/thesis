\chapter{Introduction}
The majority of this thesis explains how we performed our experimental measurement.  The role of this introductory chapter is to provide an explanation of why we performed this measurement, and what exactly it is that we are measuring.  First, a brief statement of what we measure is given.  Then, in order to understand why we perform the current measurement, we look back historically at the parallel development of nuclear structure in theoretical and experimental aspects.  After obtaining a historical perspective, it is evident why the current study is needed, and the scope of the present measurement will be re-stated.  Finally, a high-level discussion of how we perform our measurement is given, leaving the details to the remainder of the document.

\section{Statement of Purpose}
This work aims to contribute to the understanding of nucleon structure in the framework of transverse momentum dependent parton distribution functions (TMDs) by measuring structure functions in semi-inclusive deeply inelastic scattering (SIDIS).  By measuring the cross section for charged pi-mesons $\pi^{\pm}$, we measure ($F_{UU,T} + \epsilon F_{UU,L}$, $F_{UU}^{\cos\phi}$, and $F_{UU}^{\cos(2\phi)}$).  By analyzing the beam spin asymmetry (BSA) we measure the ratio

\begin{equation}
  A_{LU}^{\sin\phi} = \sqrt{2\epsilon(1-\epsilon)} \frac{F_{LU}^{\sin\phi}}{F_{UU,T}+\epsilon F_{UU,L}}
\end{equation}

for positively charged k-mesons.  Finally, we are able to use these structure functions to estimate the model parameters in TMD models.

\section{Historical Development of Nucleon Structure from Experiment and Theory}
Enlightening.

\section{Measurement of Semi-Inclusive Deeply Inelastic Scattering with CLAS}
Our purpose is re-stated clearly.

\section{Overview of our Measurement}
Let's talk about detectors, accelerators, and software written in c++. 


\chapter{Introduction}
The majority of this thesis explains how we performed our experimental measurement.  The role of this introductory chapter is to provide an explanation of why we performed this measurement, and what exactly it is that was measured.  First, a brief statement of the measurement is given.  Then, in order to understand why we perform the current measurement, a historical look back at the parallel development of nucleon structure in theoretical and experimental aspects is presented.  After obtaining a historical perspective, it is evident why the current study is needed, and the scope of the present measurement is re-stated.  Finally, a high-level discussion of how this measurement was performed is presented, leaving the details to the remainder of the document.

\section{Statement of Purpose}
This work aims to contribute to the understanding of nucleon structure in the framework of transverse momentum dependent parton distribution functions (TMDs) by measuring structure functions in semi-inclusive deeply inelastic scattering (SIDIS).  By measuring the cross section for charged pi-mesons $\pi^{\pm}$, we measure ($F_{UU,T} + \epsilon F_{UU,L}$, $F_{UU}^{\cos\phi}$, and $F_{UU}^{\cos(2\phi)}$).  By analyzing the beam spin asymmetry (BSA) we measure the ratio

\begin{equation}
  A_{LU}^{\sin\phi} = \sqrt{2\epsilon(1-\epsilon)} \frac{F_{LU}^{\sin\phi}}{F_{UU,T}+\epsilon F_{UU,L}}
\end{equation}

for positively charged k-mesons.  Finally, we are able to use these structure functions to estimate the model parameters in TMD models.

\section{Historical Development of Nucleon Structure from Experiment and Theory}

The modern picture of an atom as a dense cluster of neutrons and protons forming a nucleus surrounded by negatively charged electrons was complete at the end of the 1930s.  The electrons were attracted to the protons in the nucleus, binding the system together.  The strong force binding the nucleus was explained by Hideki Yukawa's pion (at the time undiscovered) exchange model in 1935 \cite{physics-yukawa}.  This quite simplistic view of the nature of matter provided an adequite description for nuclear physics until after World War II, when the use of bubble/spark chambers allowed for the discovery of a multitude of particles.  \\

Around the same time (1958) Hofstadter used a beam of 188 MeV electrons to bombard protons and measured the differential cross section with respect to the scattered electron angle $\theta$.  The expectation for a point-like proton (given by the Rosenbluth formula) was significantly lower than the observed cross section at large scattering angles ($\theta > 90^\circ$).  This departure suggested that the proton had a finite size, and represented an important step toward the discovery that nucleons are not fundemental particles. \\

In the early 1960s Gell-Man and Zweig developed the quark model (containing the eightfold way) that allowed the categorization of the many observed particles in terms of combinations 3 flavors of quarks \cite{physics-zweig}.  In 1965 Nambu and Greenburg demonstrated the need for a new degree of freedom called color by applying Pauli's exclusion principle to the $\Delta^{++} \; (uuu)$.  By this time, it was clear that the proton and neutron were not fundamental particles, but the idea of quarks was not widely accepted.  \\

Feynman and Bjorken suggested a very useful picture of known as the parton model in 1969 \cite{physics-feynman-1969, physics-bjorken}, which allowed the study of nucleon structure using the established methods of QED.  Feynman argued that high energy electron-proton collisions can be thought of as an electromagnetic interaction between the electron and one of the partons which make up the nucleon.  If this interaction takes place at sufficient energy $Q^{2} \approx 2 \; GeV^2/c^2$, this assumption is valid.  Using these simple ideas, Feynman and Bjorken were able to write down expectations for the electron-proton inclusive (just detecting the scattered electron) cross sections in terms of structure functions $F_{1}(x, Q^2), F_{2}(x, Q^2)$.  Bjorken predicted that the structure functions should depend only weakly on $Q^2$, and this phenomenon known as Bjorken scaling was observed shortly after.  \\

Something about PDFs.

\section{Measurement of Semi-Inclusive Deeply Inelastic Scattering with CLAS}
Our purpose is re-stated clearly.

\section{Overview of our Measurement}
Let's talk about detectors, accelerators, and software written in c++. 


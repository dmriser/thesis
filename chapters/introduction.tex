\chapter{Introduction}

In the spring of 1958, the scientists thought that the proton was a point-like particle (having mass, but no spatial extent).  This became a testable hypothesis with the ability to accelerate electrons and let collide them into nearly stationary protons, in 1948.  Moreover, the theoretical tools of the day were sufficiently advanced to predict what the electron-proton scattering cross section should be, assuming that they interact electromagnetically as point particles.  That spring, Dick Hoftstader setup the experiment in his laboratory in Dickville Delaware and his findings would create the field of nucleon structure research, a field that remains very active today.

[Insert figure from Hoftstader here]

Hoftstader observed a marked deviation of the cross section from that predicted by theorists, instead of the number of scattered electrons decreasing as the scattering angle decreased according to the inverse square, it decreased like the cube (shown in the figure above).  We now know that Hoftstader was observing the scattering of electrons from the charged quarks which compose it, and discovering that the proton is not a point-like fundamental particle.

The Hoftstader results left physicists with a clear conclusion, there is something inside of the proton, but what?  To develop a hypothesis that explains what type of particle or particles protons and neutrons are made of would take ten years of further experimentation.  During that time particle accelerators increased in energy, until collisions became so energetic that new particles were produced.  At first, physicists interpreted this "particle zoo" as all new fundamental particles.  In 1966 Murray Gell-Mann and Somebody Somewhere explained this proliferation of particles as combinations of three more fundamental particles.  The idea that a complicated set of observed particles could be reduced to a simpler basis set of fundamental particles introduced a welcome simplification, but had introduced fractionally charged particles. Despite the large volume of scattering experiments taking place, fractionally charged particles had never been detected, and still haven't today.  The un-escapable conclusion was that these new "quarks" are never found alone, but always in pairs or groups.

The results of the Hoftstader experiments could now be interpreted in terms of these new fundamental quarks. Physicists then began asking the fundamental questions which define the sub-field of nucleon structure.  How do quarks bind together to create the proton and neutron, what is the nature of the force?  What does the spatial distribution of quarks look like inside of the proton?  What about the momentum, if three quarks exist inside the proton do they share momentum equally? How can the properties of the proton (mass, spin, magnetic moment) be explained in terms of these quarks?

Unsurpringly, physicists decided to do more electron scattering experiments to probe these questions, and a useful theoretical model was provided by Richard Feynman in 1969.  Feynman's "parton model" stated that during high energy electron-proton collisions the electron enters with so much energy that the quarks which compose the proton are essentially stationary, and the electron simply scatters off of one part of the nucleus.

In this "parton" model, the cross section for electron-proton scattering was assumed to be the product of an elastic scattering between a charged electron and quark (calculable using techniques from quantum electrodynamics), and a function that describes the state of the quark in the proton prior to the collision, called the parton distribution function (PDF).  Parton distribution functions are typically indexed with the variable x that represents the fractional proton momentum carried by the struck quark.  This type of scattering is known as deeply inelastic scattering, and could be performed at the Stanford Linear Accelerator (SLAC), which powered on in 1966.

Without assuming the parton model assumption, the cross section from deeply inelastic electron-proton scattering can be parametrized in terms of structure functions, introduced in 1967 by Bjorken.  The parton model framework allowed for the calculation of these observable structure functions in terms of the PDFs mentioned above.  During the early 1970s, the predicted Bjorken scaling was observed at SLAC, and confidence in the validity of the parton model approach began increasing.

In the twenty year period between 1970 and 1990, PDFs were mapped out and an interesting picture of the momentum structure of quarks in the nucleon began to appear.  At the end of the 1990s, physicists concluded that a significant amount of the total internal energy of a proton was carried by very low momentum sea quark/anti-quark pairs that spontaneously appear and disappear continuously from the vacuum.  This interesting revelation is shown in the figure below, from [source of figure].

[Insert figure from scaling, insert figure from PDF fitters]

A small crisis occurred in 1991 when the European Muon Collaboration performed measurements of the polarized PDFs g1 and g2, concluding that the majority of the spin of the proton is not carried by the valence quarks.  Spurred on by the idea that the quarks perhaps carried some orbital angular momentum which contributed to the proton spin, it was clear that despite the success of co-linear PDFs in describing the momentum structure of the proton and neutron, a three dimensional understanding of the quark distributions in space and momentum would be helpful.

In the early 1990s it was realized by Mulders that by detecting a hadron in the final state of an electron-proton deeply inelastic scattering event (a reaction known as semi-inclusive deeply inelastic scattering), one could learn about the quark momentum in the plane transverse to the hard scattering direction along which the momentum fraction x is defined.  In an analogous manner to the DIS electron-proton cross section, the cross section for SIDIS can be decomposed into a set of 16 structure functions (depending on the polarization direction of the electron and proton). 

Again, in analogue with the simple co-linear case parton model-like assumptions are made about the nature of the scattering and all 16 of these structure functions can be calculated in terms of TMD PDFs and TMD Fragmentation Functions (TMD FFs).  The new assumption, known as factorization, is at the core of the applicability of the TMD interpretation to results of SIDIS (and other types such as Drell-Yan scattering) experiments.  The hadronic matrix element that describes the transition from proton to all possible final states is decomposed using the operator product expansion (OPE) and factorization has been proven for the SIDIS process at leading order in the expansion (twist two) [reference here].

What results from this expansion is the set of non-perturbative TMD PDF and TMD FF functions, at twist two in the expansion there are 9 such functions, at twist three there are 16.  By comparing kinematic dependence, these functions can be associated with structure functions in the cross section decomposition.

Starting in the early 2000s, collaborations  (HERA, HERMES, references) began measuring these structure functions and one pronounced effect came to be known as the Sivers effect, which is associated with the Sivers function, a twist two TMD.  After observing the sizable Sivers asymmetry, it was clear that the TMD framework has explanative power and validity.  Up to the present time, measurements of TMD PDFs and TMD FFs have endured as pilot experiments (a full review of the current situation was given in [reference to TMD review from experiment]).

In this thesis I present two distinct measurements of SIDIS structure functions and their combinations.  The first observable measured is the unpolarized cross section, which is presented as an extension of the analysis of Nathan Harrison who measured the unnormalized cross sections for charged pions in CLAS [reference to Nate].  Three structure functions appear in this cross section expansion, those which have the unpolarized beam and target subscripts "UU" in the cross section equation.  The leading structure function is not modulated by the phi dependent angle of the outgoing hadron, and is expected to be dominant in size when compared with the two co-sinusoidal structure functions that remain.  This novel measurement compliments the unnormalized work of Nathan Harrison as well as the measurements of multiplicities performed by the HERMES and COMPASS collaborations [reference both of those].  Without applying the TMD framework, this measurement still has value in that it provides the scale of the structure functions for the first time and can be used to predict SIDIS yeilds for future studies.  Within the TMD framework, the leading order unpolarized structure function is composed of the unpolarized PDF and the unpolarized fragmentation function, both fairly well known from previous experiments.  The phi modulated structure functions contain a TMD PDF known as the Boer-Mulders function, pursued for its possible connection to quark orbital angular momentum.  I expect that the results of this measurement can be used as input, with results from other experiments, to extract the Boer-Mulders TMD using a phenomenological model.

I also present measurements of positively charged kaon beam spin asymmetries, which addresses several interesting areas of TMD physics.  The beam spin asymmetry, defined as the difference in events observed for different electron helicity states (normalized by the total number of electrons), depends on the observable structure function (F LU SIN PHI).  Interestingly, if one assumes that twist three PDFs and FFs are zero, the beam spin asymmetry term vanishes because the structure function in question is composed of pure twist three products of PDFs and FFs.

Recently CLAS reported the non-zero BSA measurement of charged and neutral pion beam spin asymmetries, which implies that in the kinematics measured by CLAS, the twist three TMD terms cannot be neglected [reference to Wes Gohn thesis].  As a natural extension of this knowledge, I analyze the same observable for the heavier kaon channel to see if the same conclusion is realized.

I have reviewed the experimental hardware that enables this thesis work in chapter 2 and summarized analytical techniques used during chapter 3.  I describe basic analysis procedures that are common to both measurements in chapter 4 before discussing particle identification in chapter 5.  I present results for the inclusive scattering cross section in chapter 6, followed by the SIDIS cross section in chapter 7.  I then present my findings for the positively charged kaon BSA measurement in chapter 8 before offering a brief summary and outlook in chapter 9. 

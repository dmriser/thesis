\chapter{Experiment}

% The details of CEBAF, CLAS, and E1-F.
Jefferson National Lab houses the continuous electron beam accelerator facility (CEBAF) which is currently capable of providing 11 GeV electron beams to three experimental end stations, and 12 GeV to a fourth.  The data that are analyzed in this study are from the run period E1-F, which occurred before the 12 GeV CEBAF upgrade, the details in this chapter describe the accelerator and CLAS detector at the time of the E1-F run period.  \\

\section{CEBAF} 
Proposed in 1982 and constructed between 1987-1997 the CEBAF accelerator at Jefferson lab was composed of a pair of linear accelerators and 9 recirculating arcs arranged in a racetrack shape \cite{hardware-leeman:2001, hardware-chao:2011}.  Originally designed to provide 4 GeV unpolarized electrons to three experimental halls, CEBAF was fitted with twin polarized electron guns, and upgraded to 6 GeV beam energy before the E1-F run period.  CEBAF was built to provide an extremely high duty factor and an average beam current of up to $200 \; \mu A$.\\

\subsection{Electron Injection \& Polarization}
CEBAF's injector provided 45 MeV electrons with 70\%-80\% average longitudinal polarization for the main accelerator.  In order to provide an apparent continuous stream of events to the detectors, electron bunches were produced at a rate of $f = 1497 \; Mhz$.  To accommodate different requests for energy and beam current simultaneously, the injector produced three interspersed bunch trains at a frequency of $f/3 = 499 \; Mhz$.  The output beam energy $E = 45 \; MeV$ was chosen so that injected electrons were \textit{sufficiently relativistic}.  In other words, when bunches of electrons at different energies simultaneously passed through the linear accelerators (LINACs), the relative phase difference (between different energy bunches) accumulated over the distance remained less than $1^\circ$. \\

During early accelerator construction, the need for polarized electrons became apparent and the final polarized electron gun was produced in 2000.  Production of polarized electrons was achieved by using twin polarized electron guns mounted at $15^\circ$ with respect to the injector beam axis.  Inside of each gun, electrons were liberated from gallium arsenide photo-cathodes by three independent diode lasers operated at a repetition rate of $499 \; Mhz$.  During polarized production, the diodes operated at a central wavelength of $850 \; nm$.  By manipulating the laser polarization using a Pockels cell, the electron  beam spin is flipped at a rate of $60 \; Hz$.  Throughout the run period, an overall phase difference could be introduced by rotation of a wave-plate.  As will be discussed in some detail later, changes in beam helicity due to wave-plate settings must be removed from the recorded data.  The final output of the polarized electron guns produced an average longitudinal polarization of 70\%-80\%.  After acceleration to $5 \; MeV$ the beam polarization was measured in the injector facility using a Mott polarimeter. \\

\subsection{Acceleration of Electrons}    
The north and south LINACs were responsible for increasing the energy of the electrons from $45 \; MeV$ up to an impressive $\approx 5.7 \; GeV$ before delivery to the experimental halls A, B, and C.  In order to achieve this each bunch of electrons passed through a total of 10 acceleration stages, 5 passes through each LINAC.  The strong electric field needed to accelerate electron bunches was confined inside of superconducting 5-cell elliptical cavities.  These cavities were machined from niobium, and operated at a temperture of $2.2^\circ \; K$.  Developed at Cornell University, the cavities were operated at $1,500 \; Mhz$ with a gradient greater than $5 \; MV/m$ and a quality factor $Q_0 \geq 3 \cdot 10^9$ (the Q factor describes the monochromaticity of the cavity and is defined $Q_0 = f_0/\Delta f$).  Each cavity is sealed inside of a cryo-unit, four such units are connected together to form an $8.5$ meter cryo-module.  Each LINAC was composed of 20 such cryo-modules connected together to increase electron energies by more than $500 \; MeV$ per pass.\\

The radio frequency (RF) that powers each cavity is sourced by water-cooled 5 kilowatt Klystrons located in groups of eight above each cryo-module.  Phase locking of each cavity with a master oscillator ensures that the difference in phase between all cavities is less than one degree. The important super conductivity was maintained by circulation of liquid helium at $2.2^\circ \k K$ produced at the on-site 5 kilo-watt helium liquification plant.  \\

After re-circulation of fives passes through each LINAC, full energy beams were delivered to the halls.  Bunches were separated using an RF separator before entering their respective experimental halls.  The beamline leading to the halls was also equipped with the ability to separate bunches before they complete five full passes delivering either three full energy beams, or two full energy beams and one lower, or one full energy beam and two of lower energy.  This capability, combined with the flexible beam polarization and beam current provided by the injector ensured that each hall could experiment at it's desired settings simultaneously.  

\section{CLAS in Jefferson Lab Experimental Hall B}
The CEBAF large acceptance spectrometer (CLAS), housed in Jefferson lab's Hall B, was used to record the E1-F dataset.  Designed to detect particles over a very large angular range, the CLAS detector covered almost the full $4\pi$ solid angle around the target region.  The detector was also designed to perform efficiently for particles with wide range of momena between 0.5 and 6.0 GeV.  Overall detector design consisted of a large superconducting magnet that produced a toroidal field (this magnet was referred to as the torus), and six ideally identifcal \textit{sectors}.  Each sector of CLAS contained an identical set of sub-systems.  After combining information from all sub-systems and running sophisticated reconstruction algorithms, complete events were able to be measured.  This capability made CLAS unique in comparison with the arm style spectrometers of halls A and C.  The major components of CLAS are listed below.

\subsection{CLAS Torus and Drift Chambers}
Torus and drift chambers. 

\subsection{CLAS Cherenkov Counter}
Chernkov Counter.

\subsection{CLAS Time of Flight Scintillator}
Time-of-flight system.

\subsection{CLAS Electromagnetic Calorimeter}
Electromagnetic calorimeter


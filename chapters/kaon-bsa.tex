\chapter{Beam Spin Asymmetry Analysis}

\section{Introduction}
Measurement of the beam spin asymmetry is carried out for the positively charged k-meson.  As discussed in the introduction, the beam spin asymmetry theoretically depends on $F_{UU,L}$, $F_{UU,T}$, $F_{UU}^{\cos\phi}$, $F_{UU}^{\cos 2\phi}$, and $F_{LU}^{\sin\phi}$.  By dividing the electron-kaon events into several bins of SIDIS kinematic variables, beam spin asymmetry measurements can be taken at different average values of the kinematic variables.  Finally, the structure function ratios $A_{LU}^{\sin\phi}$, $A_{UU}^{\cos\phi}$, and $A_{UU}^{\cos 2\phi}$ can be extracted from each bin.  In this chapter, we discuss the selection of SIDIS events, the binning used in this analysis, our measurement with associated systematic uncertainties, and the extraction of structure function ratios using the $\phi$ dependence in each kinematic bin.

\section{Event Selection and Binning}
\subsubsection*{Event Selection}
After particle identification is performed on the event, those events which have a trigger electron and a positive kaon are kept for analysis.  We discard events that do not have $W > 2$ and $Q^2 > 1$, because they are not conisdered to be part of the deeply inelastic region.  Additionally, to avoid exclusive resonances in the $ep \rightarrow eK^+X$ spectrum, the authors impose a cut on the missing mass of the final state $X$.  For this analysis we use $M_{X} (ep \rightarrow eK^+X) > 1.25$.  Finally, we attempt to perform our measurement in the current fragmentation region where factorization has been proven.  This is done by excluding events with $z_h < 0.25$.  We also require that $z_h < 0.75$ to avoid exclusive events.  This restriction on $z_h$ is not applied to the $z_h$ axis, where we measure across the entire experimentally observed range.  After these selection criteria have been applied, the data is sorted into kinematic bin.

\easyFigure{image/plots/kaon-bsa/z-pt2.pdf}{Correlation between $z_h$ and $P_{T}^{2}$ for each event in our analysis sample.} 

\subsubsection*{Binning}

\easyFigure{image/plots/kaon-bsa/binning.pdf}{The binning used for each of the kinematic axes.}
For this study, the authors chose to measure the integrated beam spin asymmetry.  This simply means that for a given axis ($P_T$ for example), the events included have all observed values of the other kinematic variables (in this example $x$, $z_h$, $Q^2$).  The axes studied are $x$, $Q^2$, $z_h$, and $P_T$.  We chose to use 12 bins in $\phi$ and 10 bins of the other kinematic variables for a total of 120 analysis bins. \\

The bins were chosen using a simple method to ensure equal statistics in each bin.  The procedure will be described using the axis $x$ as an example.  First, all events are sorted by their $x$ value from smallest to largest.  Then, the smallest and largest values are recorded, which are just $x_1$ and $x_N$ if there are N events in the sample.  Next, the target number of bins M is chosen (this choice is done by the analyst based on what he/she believes to be the best choice).  Finally, the limits of each bin can be chosen simply by calculating the number of events per bin $N/M$ and then using the value of $x$ which corresponds to multiples of $N/M$ in the sample.    

\begin{equation}
  \vec{b} = (x_1, x_{N/M}, x_{2N/M}, ..., x_N)
\end{equation}

Here, the symbol $\vec{b}$ denotes a vector of (M+1) $x$ values which represent bin limits.  The binning in $\phi$ is chosen to be regularly spaced between -180 and 180 degrees.

\section{$\phi_h$ Distributions}
\subsection*{Measured Asymmetry Values}
In each bin $i$ the beam spin asymmetry (here $A_i$) is calculated according to, 

\begin{equation}
  A_i = \frac{1}{P_e} \frac{n_i^+ - n_i^-}{n_i^+ + n_i^-}
\end{equation}

where $P_e$ is the average beam polarization over the dataset.  The symbols $n_{i}^{\pm}$ refer to the number of events counted in bin $i$ with helicity $\pm$.  

\easyFigure{image/plots/kaon-bsa/bsa_x_sys.pdf}{The $\phi_h$ dependence is shown for each bin of x, increasing in value from the top left to the bottom right.  The statistical uncertainty is shown as black error bars on each point.  The total systematic uncertainty is shown as a red bar centered at zero.}

\subsection*{Statistical Uncertainties}
The uncertainty on the measured value of $A_i$ can be attributed to statistical uncertainty on the counts $n_{i}^{\pm}$, and the uncertainty associated with the measurement of $P_e$.  The treatment of the statistical uncertainty reported on the measurement includes the contribtion from counts, but not from the uncertainty in $P_e$ which is included in the systematic errors.  The uncertainty in a measured observable $\mathcal{O}$ depends on the uncertainty of the parameters used to construct it $\vec{\theta}$ in the following way (see appendix).

\begin{equation}
  \label{eqn:error-propagation}
  \sigma_{\mathcal{O}}^2 = \sum_{i=1}^{N} \sum_{j=1}^{N} \frac{\partial \mathcal{O}}{\partial \theta_i} \frac{\partial \mathcal{O}}{\partial \theta_j} \rho_{ij} \sigma_i \sigma_j 
\end{equation}
  
For the beam spin asymmetry in the $i^{th}$ bin $A_i$ one finds that without correlations ($\rho_{ij} = \delta_{ij}$) the error propagation proceeds as shown below.

\begin{equation}
  \sigma_{A}^{2} = \frac{A^2}{P_{e}^2} \sigma_{P_{e}}^{2} + \frac{4 (n_{-}^{2} \sigma_{+}^{2}  + n_{+}^{2} \sigma_{-}^{2})}{ P_{e}^{2} (n_{+} + n_{-})^4}
\end{equation} 

The first term which is the contribution from the variance in the measurements of beam polarization will be included as a systematic error.  The second term is used as the statistical error bars shown through the analysis.  The counts $n_{\pm}^{i}$ for the $i^{th}$ bin are assumed to be Poisson in nature, and therefore have a variance equal to the expected number of counts $\sigma_{\pm}^{2} = n_{\pm}^{i}$.  With this expression for the statistical uncertainty on the counts, and dropping the beam polarization term for now, the expression becomes: 

\begin{equation}
  \sigma_{A}^{2} = \frac{4n_+ n_-}{P_{e}^{2} (n_+ + n_-)^3}
\end{equation}


\easyFigure{image/plots/kaon-bsa/bsa_z_sys.pdf}{The $\phi_h$ dependence is shown for each bin of $z_h$, increasing in value from the top left to the bottom right.  The statistical uncertainty is shown as black error bars on each point.  The total systematic uncertainty is shown as a red bar centered at zero.}

\subsection*{Systematic Uncertainties}
\subsubsection*{Formalism}
Systematic effects are shifts or biases in the measured result of some observable as a result of the procedure used in the measurement.  Systematic effects can typically be identified and corrected for, or removed all together from the measurement.  In the cases where an effect cannot be completely removed, the degree to which the correction for the effect is uncertain is included in the result of the measurement as a systematic uncertainty. \\

Sources of systematic effects can include background events from different processes which enter the sample, calibrations of different detector systems, misalignments in detector geometry, and biases in selection criteria.  Each of the systematic sources mentioned here has at least one associated procedure for correcting it's effect on the analysis.  As an example consider momentum corrections in CLAS.  These corrections are performed to remove the effect of slight mis-alignments in detector geometry from what is in reconstruction, as well as slight differences between the true magnetic field and the field map used in reconstruction.  These physical effects introduce a systematic effect, the particle 4-momenta reconstructed are shifted away from the true values.  Standard reactions (elastic scattering) can be used to develop corrections for the 4-momenta of particles, and these corrections typically depend on a set of parameters $\vec{\theta}$, which have an associated parameter uncertainty described by a covariance matrix $V_{ij}$.  It is these parameter uncertainties that propagate through to the final observables, and the assignment of the magnitude of such effects is then what is refered to as systematic uncertainty. \\

\easyFigure{image/diagrams/linear-error.pdf}{The analysis is run for variations in the input parameters $\theta_i$ to calculate the dependence of the result $\mathcal{O}$ on each parameter, as described in this section.}

Systematic uncertainties can be included using the standard equation for error propagation.  In some cases it is possible to analytically find the derivatives needed to calculate the dependence of the observable on a source of systematic uncertainty.  This is the case for effect of the variance of the beam polarization on the beam spin asymmetry observable.  However in many cases, it is not possible to analytically calculate the effect of some analysis parameter $\theta_i$ on the observable $\mathcal{O}$.  Since the observable is usually calculated using some computational chain which starts with the input parameters $\vec{\theta}$, it is possible to find the dependence of the observable $\mathcal{O}$ on the inputs numerically.

\begin{equation}
  \frac{\partial \mathcal{O}}{ \partial \theta_i} \approx \frac{\mathcal{O}(\theta_i + \sigma_{\theta_i}/2) - \mathcal{O}(\theta_i - \sigma_{\theta_i}/2)}{\sigma_{\theta_i}}
\end{equation}

After inserting the above into equation \ref{eqn:error-propagation} one finds,

\begin{equation}
  \sigma_{\mathcal{O}}^{2} = \sum_{i=1}^{n} \sum_{j=1}^{n} \rho_{ij} (\mathcal{O}(\theta_i + \sigma_{\theta_i}/2) - \mathcal{O}(\theta_i - \sigma_{\theta_i}/2)) (\mathcal{O}(\theta_j + \sigma_{\theta_j}/2) - \mathcal{O}(\theta_j - \sigma_{\theta_j}/2)) 
\end{equation}

where $\rho_{ij}$ is the correlation $V_{ij}/\sigma_i \sigma_j$.  In most cases, these correlations are assumed to be zero.  In some cases, when the parameters $\theta_i$, $\theta_j$ come from a fit one may have a correlation provided by the covariance matrix and it should be used.  In the case where correlations are assumed to be zero, the total systematic uncertainty is simply the quadrature sum of the shift in the observable within the uncertainty window on each parameter.

\begin{equation}
  \sigma_{\mathcal{O}}^{2} = \sum_{i=1}^{n} \Bigl[ \mathcal{O}(\theta_i + \sigma_{\theta_i}/2) - \mathcal{O}(\theta_i - \sigma_{\theta_i}/2) \Bigr]^2
\end{equation}

Another approach exists that takes into account possible correlations between the analysis parameters $\theta_i$.  This approach has not yet been widely used, and probably requires a thorough understanding of systematics using the previously described method before it's application.  The approach consists of generating monte carlo $M$ sets of parameters $\vec{\theta}$ and obtaining $M$ results for the observable $\mathcal{O}$.  The results are then interpreted probabalistically and the observable value and total systematic error are reported as the mean and standard deviation of the results.

\begin{gather}
  \expval{\mathcal{O}} = \frac{1}{M} \sum_{i=1}^{M} \mathcal{O}_i \\
  \sigma_{\mathcal{O}}^{2} = \frac{1}{M-1} \sum_{i=1}^{M} (\mathcal{O}_i - \expval{\mathcal{O}})^2
\end{gather}

\subsubsection*{Sources of Systematic Uncertainty}
The table \ref{table:kaon-systematics} below summarizes the sources of systematic effects considered in this analysis.

% --------------------
%    table of sys 
% --------------------
\begin{table}
  \centering
  \begin{tabular}{c|c|c}
    Source                     & Variation & Magnitude \\ 
    \hline
    Beam polarization          & 0.024          & 0.000672 \\ 
    DC Region 1 Fid.           & 1 (cm)         & 0.001344 \\ 
    DC Region 3 Fid.           & 3 (cm)         & 0.001821 \\
    EC-W                       & 12 (cm)        & 0.000948 \\ 
    EC-V                       & 12 (cm)        & 0.000797 \\
    EC-U                       & 12 (cm)        & 0.002487 \\
    Kaon Confidence ($\alpha$) & 0.01-0.07      & 0.001827 \\
    $\theta_{cc}$ Matching     & $\sigma$       & 0.001152 \\
    EC Energy Deposition       & 0.01 (GeV)     & 0.001644 \\
    $p_{K^+}$                  & 2.5-$E_{beam}$ & 0.002360 \\ 
    EC Sampling Fraction       & $0.5 \sigma$   & 0.001240 \\
    Z-Vertex                   & 0.5 (cm)       & 0.002581 \\
    \hline 
    Statistical                & -              & 0.007494 \\ 
    \hline
    MC Estimate                & -              & 0.002917
  \end{tabular}
  \caption{Different sources of systematic effect considered in this analysis.  The magnitude of the effect is shown here averaged over all bins.  The units of the shift are just the same units of the asymmetry value. }
  \label{table:kaon-systematics}
\end{table}

% Do we need this figure?
%\easyFigure{image/plots/kaon-bsa/systematic-bar-x.pdf}{The fractional contribution to the total systematic uncertainty by each source, averaged over the bins in the x axis.}

Except for the beam polarization and the momentum of the kaon track, all parameters listed in the table are treated using the formalism outlined above.  The beam polarization uncertainty quoted at 2.4\% contains contributions from the standard deviation of the Moller polarimetry measurements (0.2\%), residual target polarization effects (1.4\%), and atomic motion/finite acceptance corrections (0.8\%).\\

Because of the inability to distinguish kaon, pion, and proton tracks at higher momentum, the maximum kaon track momentum is varied between 5.5 (no maximum) and 2.5 GeV.  The difference between these results is quoted as a systematic uncertainty and added in quadrature with the other sources.  This source of systematic uncertainty has a larger effect on the $z_h$-axis, because $z_{max}$ is limited by limiting $p_{max}$.  While for the large $z_h$-bins this contribution is dominant, its size is comparable with other systematic sources throughout the remainder of the bins.

\section{Extraction of Modulations}
The motivation to measure the beam spin asymmetry in several kinematic bins as well as bins of $\phi_{h}$ is to perform an estimate of the value of structure functions at the kinematic points (or the average value of the structure functions over the range of values included in a point).  To do this, the authors perform parameter estimation on the $\phi_{h}$ distributions taking as a model the theoretical dependence of the beam spin asymmetry on $\phi_{h}$.

\begin{equation}
  f(\phi_h, \vec{a}) = \frac{a_0 \sin\phi_h}{1 + a_1 \cos\phi_h + a_2 \cos(2\phi_h)}
\end{equation}

The parameters $\vec{a}$ are the structure function ratios we wish to extract.  The simplest way to extract these parameters is to use a standard fitting package like \texttt{Minuit} or \texttt{scipy.optimize.minimize}.  In these approaches, $\chi^2$ is defined as the square difference between the observed data values and those predicted by the model, normalized by the error.  If the fluctuation between the data and theory predictions is on the order of the error, the $\chi^2$ is simply on the order of the number of data points.  The parameters $\vec{a}$ which best describe the data are those which make the $\chi^2 (\vec{a})$ assume it's minimum value.  This minimization is done in practice with gradient descent or quasi-Newtons method based algorithms, and we will not discuss details of these here.  It is sufficient to say that these methods produce the parameters $\vec{a}$, and an estimate of the covariance matrix $V$.  These parameters and their errors become the extracted value and uncertainty on the extracted values of the structure function ratio in each bin. \\

Unfortunately, applying the standard single-fit procedure described above does not always produce stable results.  In some cases, the resulting parameter sets are reasonable, in other cases however the parameters in the denominator become unphysically large and oppose each other.  This effect has motivated previous analysts to search for other means of extracting the dominant $\sin\phi_h$ behaviour from the distributions.  One common technique is to assume that $a_1$ and $a_2$ of above are small compared to 1.  The analyst can then fit the $\phi_h$ distribution with just one linear parameter $a_0$.  This produces a stable result, but has the disadvantage that one needs to introduce a systematic uncertainty associated with the difference observed between using the full model (with a restricted range for the parameters in the denominator) and the results obtained using the single parameter model.  In order to avoid this, the authors choose to use a Monte Carlo method of replicas.  The replica method consists of generating $N_{rep}$ psuedo-data $\phi_h$ distributions which have a normal distrubition located at the observed value, and with a variance equal to the statistical errors on the associated data point.  

\begin{equation}
  \vec{A}_{rep} = \mathcal{N}(\vec{A}, \vec{\sigma_{A}})
\end{equation}

Where here $\vec{A}$ is a vector of length $n_{phi}$ bins, representing the measured beam spin asymmetry for each value of $\phi_h$ in a given kinematic bin.  Each of these distributions is fit with the full model, and the resulting parameter values are saved.  The final reported value for each fit parameter, as well as it's uncertainty can be reported as the mean, and standard deviation of the fit results.  This procedure which is similar to bootstrapping, can be seen as an attempt to fit the underlying distribution that generated the data while avoiding the statistical noise.  This technique has been discussed in \cite{computing-watt:2012}.

\begin{gather}
  \expval{a_j} = \sum_{i=1}^{N_{rep}} a_{j}^{(i)} \\
  \sigma_{a_j}^{2} = \frac{1}{N_{rep}-1} \sum_{i=1}^{N_{rep}} (a_{j}^{(i)} - \expval{a_j}) 
\end{gather}

\easyFigure{image/plots/kaon-bsa/p-study-x.pdf}{The BSA for each bin of $x$, $\phi_h$ (all plotted together on the x-axis) is compared with and without a maximum momentum (no tracks exceed 5 GeV) for the kaon track.  The global bin coordinate on the x-axis is $i + n_{\phi} * j$ where $i$ is the $\phi_h$ bin, $j$ is the $x$ bin, and $n_{\phi}$ is the number of $\phi$ bins (12). Both $i$ and $j$ start at 0.}

\subsubsection*{Results}
We observe that the term $A_{LU}^{\sin\phi}$ is weakly dependent on the kinematic variables.  The value is around +3\% for most bins.  

\easyFigure{image/plots/kaon-bsa/alu_sin.pdf}{Our extraction of $A_{LU}^{\sin\phi}$ for the kinematic bins described above.  The black error bars represent uncertainty in the extraction of the parameter value.  Red error bars are systematic uncertainties.}

\chapter{Particle Identification}

% -----------------------------------------
%    chapter motivation and purpose 
% -----------------------------------------

\section{Introduction}
Particle identification (PID) is the process of classifying tracks as known particles.  After reconstruction and matching of detector responces to each track, the reconstruction package \texttt{recsis} assigns a preliminary particle identification based on loose selection criteria.  In this analysis, tracks are classified based on a more stringent criteria.  This chapter discusses the methodology used by the authors to classify particles.

\section{Electron Identification}
Electrons in CLAS are abundant, and the detection of an electron is a basic necessity for every event that will be analyzed.  Each negative track is considered a possible electron, a series of physically motivated cuts is applied.  If a track passes all cuts, it is identified as an electron.  All track indices which pass electron identification are saved, and the one with the highest momentum is used in the analysis.  
\\


\subsection{Electron ID Cuts}

The cuts used to select electrons are enumerated below.

\begin{itemize}
  \item{Negative charge}
  \item{Drift chamber region 1 fiducial}
  \item{Drift chamber region 3 fiducial}
  \item{Electromagnetic Calorimeter fiducial (UVW)}
  \item{EC minimum energy deposition}
  \item{Sampling Fraction (momentum dependent)}
  \item{z-vertex position}
  \item{Cherenkov counter $\theta_{cc}$ matching to PMT number}
  \item{Cherenkov counter $\phi_{rel}$ matching to PMT (left/right)}
\end{itemize}

Each cut will now be described in more detail.

\subsubsection*{Negativity Cut}
Each track is assigned a charge based on the curvature of it's trajectory through the magnetic field of the torus.  This is done during the track reconstruction phase.  The tracks are eliminated as electron candidates if they are not negatively charged.

\subsubsection*{Drift chamber fiducial}
Negative tracks which pass geometrically close to the edges of the drift chamber are, from a tracking perspective, more difficult to understand.  Often these tracks originate from downstream background, or are otherwise unacceptable.  Additionally, tracks which fall outside of the fiducial region of the drift chambers are likely to fall outside of the fiducial region of the downstream detectors as well.  For these reasons, it is common to remove tracks which are geometrically close to the boundaries of the drift chambers in region 1 as well as region 3 coordinate systems.

\subsubsection*{Electromagnetic Calorimeter fiducial (UVW)}
As tracks traverse the electromagnetic calorimeter they develop electromagnetic showers.  If the track passes close to the edges of the detector, there is a chance that the shower will not be fully contained within the calorimeter volume (it spills out the edges).  For this reason, it has become standard to remove the hits which fall within the outer 10 centimeters of each layer of the EC (10 centimieters is the width of a scintillator bar).  This cut is applied in the U, V, W coordinate system.  

\easyFigure{image/plots/electron-id/ec-fid.png}{All negative tracks are shown here in black.  In color, the tracks which pass the EC fiducial cut are shown.}

\subsubsection*{EC minimum energy deposition}
The negative tracks that start out as electron candidates are primarily composed of electrons and negative $\pi$ mesons.  One way to differentiate between these two species is to exploit the difference in energy deposition between the two in the electromagnetic calorimeter.  Electron typically develop a much larger more energetic shower than $\pi$ mesons, which minimally ionize the calorimeter material.  The result is that the total energy deposition is typically larger for electrons than $\pi$ mesons.  In this analysis we require that at least 60 MeV was deposited in the inner calorimeter for electron candidates.  

\subsubsection*{Sampling Fraction (momentum dependent)}
The electromagnetic calorimeter is designed such that electrons will deposit $E_{dep}/p \approx 0.3$ approximately one-third of their energy, regardless of their momentum.  In contrast to this, the ratio $E_{dep}/p$ for $\pi$ mesons decreases rapidly with momentum.  To develop a momentum dependent cut for this distribution, all negative candidates are first filled into a two-dimensional histogram of $E_{dep}/p$ vs. $p$.  The histogram is then binned more coarsely in momentum, and projected into a series of 40 slices.  Each of these slices is fit with a Gaussian to extract the position $\mu_i$ and width $\sigma_i$ of the electron peak.  Finally, the authors choose a functional form for the mean and standard deviation of the distributions to be a third order polynomial in momentum.

\begin{eqnarray}
  \mu (p) = \mu_0 + \mu_1 p + \mu_2 p^2 + \mu_3 p^3 \\
  \sigma (p) = \sigma_0 + \sigma_1 p + \sigma_2 p^2 + \sigma_3 p^3 
\end{eqnarray}    

Boundaries are constructed from this information by adding/subtracting $n_{\sigma}$ from the mean.  In the nominal case, we use $n_{\sigma} = 2.5$.

\begin{eqnarray}
  f_{max} (p) = \mu (p) + n_{\sigma} \sigma (p) = (\mu_0 + n_{\sigma} \sigma_0) + (\mu_1 + n_{\sigma} \sigma_1)p + (\mu_2 + n_{\sigma} \sigma_2)p^2 + (\mu_3 + n_{\sigma} \sigma_3)p^3 \\
  f_{min} (p) = \mu (p) - n_{\sigma} \sigma (p) = (\mu_0 - n_{\sigma} \sigma_0) + (\mu_1 - n_{\sigma} \sigma_1)p + (\mu_2 - n_{\sigma} \sigma_2)p^2 + (\mu_3 - n_{\sigma} \sigma_3)p^3
\end{eqnarray}

Due to slight differences between the 6 sectors of the CLAS detector, the authors choose to calibrate and apply this cut for each sector individually.  The results are shown in table \ref{table-sampling-fraction}.

% ---------------------------------
%  table of cut values for this 
% ---------------------------------

\begin{table}[h]
  \centering 

  \begin{tabular}{c | c | c | c | c | c | c}
    Parameter & Sector 1 & Sector 2 & Sector 3 & Sector 4 & Sector 5 & Sector 6                           \\
    \hline
    $\mu_3$     & -8.68739e-05 & 0.000459313  &  9.94077e-05 & -0.000244192 & -7.65218e-05 & -0.000392285  \\
    $\mu_2$     & -0.000338957 & -0.00621419  & -0.00267522  & -0.00103803  & -0.00222768  & -0.00105459   \\
    $\mu_1$     &  0.0191726   &  0.0393975   &  0.02881     &  0.0250629   &  0.0233171   &  0.0265662    \\
    $\mu_0$     &  0.2731      &  0.296993    &  0.285039    &  0.276795    &  0.266246    &  0.25919      \\
    $\sigma_3$  & -0.000737136 &  0.000189105 & -0.000472738 & -0.000553545 & -0.000646591 & -0.000633567  \\
    $\sigma_2$  &  0.00676769  & -0.000244009 &  0.00493599  &  0.00434321  &  0.00717978  &  0.00626044   \\
    $\sigma_1$  & -0.0219814   & -0.00681518  & -0.0180929   & -0.0140827   & -0.0246181   & -0.022029     \\
    $\sigma_0$  &  0.0474188   &  0.0475098   &  0.0461743   &  0.0492728   &  0.0546257   &  0.0517508    
  \end{tabular}
  \caption{$\mu$ and $\sigma$ values used to construct the momentum dependent sampling fraction cut.}
  \label{table-sampling-fraction}
\end{table}

\subsubsection*{z-vertex position}
Electrons can be produced as part of $e^+ e^-$ pairs.  For this analysis, these are not of interest.  The authors choose to select only electrons which originate from the target and are believed to be the scattered incoming electron.  For this reason the authors accept only electron candidates which have a z-vertex $v_z \in [-27.7302, -22.6864]$.  This cut is applied after the vertex position has been corrected (this correction will be discussed in a subsuquent chapter).

\subsubsection*{Cherenkov counter $\theta_{cc}$ and $\phi_{rel}$ matching to PMT}
The placement of photo-multiplier tubes (PMT) in the Cherenkov counter allows for additional consistency conditions to be applied.  The placement of 18 PMTs increasing in polar angle away from the beamline means that the PMT segment number is correlated to the angle which the electron has with the beamline at the Cherenkov counter $\theta{cc}$.  Additionally, PMTs that are placed on the left and right of the detector can be used to check consistency with the azimuthal angle the track forms with the central line of the detector (ie $\phi_{rel} > 0$ means the track was in the right half of the sector, $\phi_{rel} < 0$ means the track was in the left half of the sector).  An integer code is used to describe the PMT associated with the track.  The left PMT is assigned value -1, the right 1, and a signal in both PMTs is assigned 0.  If both PMTs have a signal, the track is allowed to pass.  If the left PMT was the one that had a signal, only events with $\phi_{rel} < 0$ are allowed to pass.  Similarly if the right PMT fired (code = 1), only events with $\phi_{rel} > 0$ are allowed to pass.  Technical note: the integers in question can be obtained from the ntuple22 format tree by doing the following.

\begin{lstlisting}
  for (int index = 0; index < event.gpart; index++){
    int pmt = event.cc_segm[index]/1000 - 1;
    int segment = event.cc_segm[index]%1000/10; 
  }
\end{lstlisting}

\easyFigure{image/diagrams/relative-phi.pdf}{The angle $\phi_{rel}$ is the azimuthal angle between the central line of the detector and the track.}


\section{Hadron Identification}


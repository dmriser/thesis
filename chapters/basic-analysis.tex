\chapter{Basic Analysis \& Corrections}

\section{Introduction}
This chapter discusses analysis procedures that are common to the subsequent data analyses of pions and kaons.  These procedures can be divided into two groups.  The first type of basic analysis described is the aggregation or calculation of scalar values over the run-period (luminosity, helicity).  The second type of analysis procedure described is a correction to measured values.  Vertex corrections, timing corrections, and kinematic corrections will be discussed.


\section{Luminosity Calculation}

\begin{equation}
	\frac{dN_X}{dt} = \mathcal{L} \sigma_X 
\end{equation}

\begin{equation}
	\mathcal{L} = j_e \rho_p l_T 
\end{equation}

\begin{equation}
	N_X = \integral_{0}^{t_{exp}} j_e \rho_p l_T \sigma_X dt = \rho_p l_T \sigma_X \integral_{0}^{t_{exp}} j_e dt =  \rho_p l_T \sigma_X \Delta Q
\end{equation}



\section{Determination of Good Run List}

\section{Helicity Determination}

\section{Vertex Corrections}

\section{Timing Corrections}

\section{Kinematic Corrections}

A network of hard working and caring people have supported and helped me through this journey.  Most directly, my supervisor Dr. Kyungseon Joo has supported me from the beginning and shared with me his valuable insights and experience.

At Jefferson lab, I had the opportunity to collaborate with world class physicists.  I would like to thank the CLAS collaboration and all of its members.  In particular, the door of Dr. Harut Avakian was always open for discussion, and I relied heavily on his expertise in analyzing and interpreting scattering data.  I have also had the pleasure of working with Dr. Youri Sharabian on the HTCC, and Dr. Maurizio Ungaro on the GEMC drift chamber geometry.  

Dr. Joo has done a great job assembling a strong research group, and I thank each member for their support in this work.  For more than a year I worked closely with Dr. Nick Markov, and his expertise in data analysis was the driving force behind the successful extraction of the inclusive cross section.  Nick makes himself accessible for any kind of question, and I have made use of his insights many times over the last four years.  For discussions of the details of c++, code design, and general computing I have found Dr. Andrey Kim an invaluable resource.  I have also enjoyed collaborating with Dr. Stefan Diehl on several small but important analysis components.  From my collaborations with Dr. Nobuo Sato I learned about statistics, global analysis, and monte carlo sampling methods.

I am grateful to the educators who have contributed to my development.  Drs. Gerald Dunne, Peter Schweitzer, Alexander Kovner, Richard Jones, Susanne Yelin, Elena Dormidontova, Juha Javanainen and Philip Mannheim of the University of Connecticut all delivered insightful presentations of physics and made the experience of coursework feel like following the path of discovery of the fundamental laws of physics for the first time.  I will forever cherish my memories of those lectures.  I was first motivated to continue to graduate school by two professors at Delaware State University.  Drs. Essaid Zerrad and Gour Pati both spent countless hours with me as an undergraduate, educating me on the basic methods of computational and experimental research and giving me a real taste of research life.  

Sharing this experience with others has created several friendships.  Brandon Clary and Dr. Frank Cao have been with me since the beginning in 2013 when we started classes at UConn together.  I have fond memories of solving problems with both of them during our first two years of classes together.  I am also thankful for Kemal Tezgin sharing so many of his lunch breaks with me walking in the forest, explaining theoretical concepts and discussing philosophy.  I have also enjoyed interacting with Dr. Freddy Obrecht, who is always working on an interesting project or has found an interesting challenge.

Ultimately, none of this would have been possible without the support of my family.  My parents have always provided me with any type of support that I needed, and have always believed in me.  

Finally, I would like to thank my wife Molly, who has selflessly followed me around the east coast for the last six years in the pursuit of my education.  Not only has she supported my goals every day of my graduate school journey, she helped me in forging good study habits as an undergraduate student as well.  Without her love, support, and companionship, this work wouldn't have been possible.


